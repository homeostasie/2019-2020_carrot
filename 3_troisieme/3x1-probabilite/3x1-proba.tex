\documentclass[12pt]{article}
\usepackage{geometry} % Pour passer au format A4
\geometry{hmargin=1cm, vmargin=1cm} % 

% Page et encodage
\usepackage[T1]{fontenc} % Use 8-bit encoding that has 256 glyphs
\usepackage[english,french]{babel} % Français et anglais
\usepackage[utf8]{inputenc} 

\usepackage{lmodern}
\setlength\parindent{0pt}

% Graphiques
\usepackage{graphicx,float,grffile}

% Maths et divers
\usepackage{amsmath,amsfonts,amssymb,amsthm,verbatim}
\usepackage{multicol,enumitem,url,eurosym,gensymb}

% Sections
\usepackage{sectsty} % Allows customizing section commands
\allsectionsfont{\centering \normalfont\scshape}

% Tête et pied de page

\usepackage{fancyhdr} 
\pagestyle{fancyplain} 

\fancyhead{} % No page header
\fancyfoot{}

\renewcommand{\headrulewidth}{0pt} % Remove header underlines
\renewcommand{\footrulewidth}{0pt} % Remove footer underlines

\newcommand{\horrule}[1]{\rule{\linewidth}{#1}} % Create horizontal rule command with 1 argument of height

%----------------------------------------------------------------------------------------
%   Début du document
%----------------------------------------------------------------------------------------

\begin{document}

%----------------------------------------------------------------------------------------
% RE-DEFINITION
%----------------------------------------------------------------------------------------
% MATHS
%-----------

\newtheorem{Definition}{Définition}
\newtheorem{Theorem}{Théorème}
\newtheorem{Proposition}{Propriété}

% MATHS
%-----------
\renewcommand{\labelitemi}{$\bullet$}
\renewcommand{\labelitemii}{$\circ$}
%----------------------------------------------------------------------------------------
%   Titre
%----------------------------------------------------------------------------------------

\setlength{\columnseprule}{1pt}

\horrule{2px}
\section*{Chapitre 1 - Hasard}
\horrule{2px}

\subsection*{1 - Vocabulaire}

\begin{Definition}{Expérience aléatoire}\\
Une expérience est aléatoire si elle vérifie deux conditions.
\begin{itemize}
\item On connait toutes les \textbf{issues} possibles. \textit{On sait ce qui peut se passer.}
\item Le résultat n'est pas \textbf{prévisible}. \textit{On ne sait pas ce qui va se passer.}
\end{itemize}
\end{Definition}

\begin{Definition}{Évènement}\\
Un évènement est un ensemble d'issues possibles.
\end{Definition}

\begin{Definition}{Probabilité}\\
La probabilité d'un évènement exprime la \og chance\fg\, de le voir se produire. C'est un nombre compris entre 0 et 1.  \textit{Ce n'est pas un pourcentage}.

\begin{itemize}
\item Si la probabilité d'un évènement est 0, alors il ne se réalise jamais. Il est \textbf{impossible}.
\item Si la probabilité d'un évènement est 1, alors il se réalise à chaque fois. Il est \textbf{certain}.
\item Si la probabilité d'un évènement est $\frac{1}{2}=0.5$, alors il se réalise en moyenne une fois sur deux. 
\end{itemize}
\end{Definition}

\subsection*{2 - Fréquences}

On peut comprendre une expérience aléatoire en la répétant un grand nombre de fois. On étudie alors les fréquences des issues. Elles sont proches des résultats théoriques. On utilise alors l'informatique.

Sur un tableur : \textit{\textbf{ALEA.ENTRE.BORNES(1;6)}} retourne un nombre entier au hasard entre 1 et 6.
Sur calculatrice : \textit{\textbf{ran}} et \textit{\textbf{ranit(1;6)}}


\subsection*{3 - Calculer des probabilités}

On n'est pas obligé d'avoir des dés, des cartes ou un ordinateur pour calculer des probabilités. On peut aussi les chercher \og à la main \fg . 

\begin{enumerate}
\item[1.] Le \textbf{dénombrement}. On compte à la main. On liste les issues possibles au brouillon et on réfléchit...
\item[2.] Le \textbf{tableau}. Très pratique pour représenter les expériences qui se répètent deux fois.
\item[3.] Les \textbf{arbres}. Très pratique pour représenter les expériences avec des enchainements.
\end{enumerate}

Du coup, pour calculer la probabilité :
$$p = \dfrac{\text{Nombre de fois que l'issue se répète}}{\text{Nombre total des issues}}$$

Remarque : Une probabilité est souvent écrite comme une fraction... il y a donc plusieurs écritures possibles.


\end{document}
