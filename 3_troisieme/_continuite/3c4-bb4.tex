\documentclass[10pt]{article}
\usepackage[T1]{fontenc}
\usepackage[utf8]{inputenc}
\usepackage{fourier}
\usepackage[scaled=0.875]{helvet} 
\renewcommand{\ttdefault}{lmtt} 
\usepackage{amsmath,amssymb,makeidx}
\usepackage[normalem]{ulem}
\usepackage{fancybox,graphics,graphicx}
\usepackage{tabularx}
\usepackage{pifont}
\usepackage{ulem}
\usepackage{dcolumn}
\usepackage{enumitem}
\usepackage{textcomp}
\usepackage{diagbox}
\usepackage{tabularx,multicol}
\usepackage{multirow}
\usepackage{lscape}
\usepackage{scratch}
\newcommand{\euro}{\eurologo{}}
%Tapuscrit : Denis Vergès
\usepackage{pstricks,pst-plot,pst-text,pst-tree,pstricks-add}
\usepackage[left=3.5cm, right=3.5cm, top=2cm, bottom=2cm]{geometry}
\newcommand{\vect}[1]{\overrightarrow{\,\mathstrut#1\,}}
\newcommand{\R}{\mathbb{R}}
\newcommand{\N}{\mathbb{N}}
\newcommand{\D}{\mathbb{D}}
\newcommand{\Z}{\mathbb{Z}}
\newcommand{\Q}{\mathbb{Q}}
\newcommand{\C}{\mathbb{C}}
\renewcommand{\theenumi}{\textbf{\arabic{enumi}}}
\renewcommand{\labelenumi}{\textbf{\theenumi.}}
\renewcommand{\theenumii}{\textbf{\alph{enumii}}}
\renewcommand{\labelenumii}{\textbf{\theenumii.}}
\def\Oij{$\left(\text{O}~;~\vect{\imath},~\vect{\jmath}\right)$}
\def\Oijk{$\left(\text{O}~;~\vect{\imath},~\vect{\jmath},~\vect{k}\right)$}
\def\Ouv{$\left(\text{O}~;~\vect{u},~\vect{v}\right)$}
\usepackage{fancyhdr}
\usepackage{hyperref}
\hypersetup{%
pdfauthor = {APMEP},
pdfsubject = {Brevet des collèges},
pdftitle = {Centres étrangers 14 juin 2019},
allbordercolors = white,
pdfstartview=FitH}    
\thispagestyle{empty}
\usepackage[frenchb]{babel}
\usepackage[np]{numprint}
\begin{document}

\section*{S5 : Semaine du 13/04 au 19/04 - Exercice complémentaire}

\begin{itemize}
  \item Brevet 2019 - Centres étrangers
  \item 16 points
  \item 15 / 35 minutes pour l'exercice
  \item 15 minutes pour la lecture et la compréhension de la correction
\end{itemize}


\textbf{Partie I}

\emph{Dans cette partie, toutes les longueurs sont exprimées en centimètre}.

On considère les deux figures ci-dessous, un triangle équilatéral et un rectangle, où $x$ représente un nombre positif quelconque.

\begin{center}
\begin{tabularx}{\linewidth}{*{2}{>{\centering\arraybackslash}X}}
    \begin{pspicture}(-2,-1)(2,2)
        \pspolygon(2;-30)(2;90)(2;210)
        \psline(0.9;30)(1.1;30)
        \psline(0.9;150)(1.1;150)
        \psline(0.9;-90)(1.1;-90)
        \uput[d](1.1;-90){$4x + 1$}
        \end{pspicture}&\psset{unit=0.9cm}\begin{pspicture}(-2,-1.5)(2,1.5)
        \psframe(-2,-1.5)(2.5,1)
        \psframe(-2,-1.5)(-1.75,-1.25)\psframe(-2,0.75)(-1.75,1)
        \psframe(2.5,-1.5)(2.25,-1.25)\psframe(2.5,0.75)(2.25,1)
        \uput[d](0.25,-1.5){$4x + 1,5$}\uput[r](2.5,-0.25){$2x$}
    \end{pspicture}\\
\end{tabularx}
\end{center}

\bigskip

\begin{enumerate}
\item Construire le triangle équilatéral pour $x = 2$.
\item  
	\begin{enumerate}
		\item Démontrer que le périmètre du rectangle en fonction de $x$ peut s'écrire $12 x + 3$.
		\item Pour quelle valeur de $x$ le périmètre du rectangle est-il égal à $18$~cm ?
 	\end{enumerate}
\item  Est-il vrai que les deux figures ont le même périmètre pour toutes les valeurs de $x$ ?
Justifier.
\end{enumerate}

\bigskip

\textbf{Partie II}

\medskip

On a créé les scripts (ci-contre) sur Scratch qui, après avoir demandé la valeur de $x$ à l'utilisateur, construisent les deux figures de la partie I.

Dans ces deux scripts, les lettres A, B, C et D remplacent des nombres.

Donner des valeurs à A, B, C et D pour que ces deux scripts permettent de construire les figures de la partie 1 et préciser alors la figure associée à chacun des scripts.

\begin{multicols}{2}

\begin{scratch}
    \initmoreblocks{définir \namemoreblocks{script 1}}
    \blocksensing{demander \ovalnum{Donner une valeur} et attendre}
    \blockpen{stylo en position d’écriture}
    \blockrepeat{répéter \ovalnum{A} fois}
    {
        \blockmove{avancer de \ovalnum{4} * réponse + \ovalnum{1,5}}
        \blockmove{tourner \turnleft{} de \ovalnum{B} degrés}
        \blockmove{avancer de \ovalnum{2} * réponse}
        \blockmove{tourner \turnleft{} de \ovalnum{90} degrés}
    }
    \blockpen{relever le stylo}
\end{scratch}

\begin{scratch}
    \initmoreblocks{définir \namemoreblocks{script 2}}
    \blocksensing{demander \ovalnum{Donner une valeur} et attendre}
    \blockpen{stylo en position d’écriture}
    \blockrepeat{répéter \ovalnum{C} fois}
    {
        \blockmove{avancer de \ovalnum{4} * réponse + \ovalnum{1}}
        \blockmove{tourner \turnleft{} de \ovalnum{D} degrés}
    }
    \blockpen{relever le stylo}
\end{scratch}

\end{multicols}

\newpage


\textbf{Partie I}

\begin{enumerate}
\item On trace un segment de longueur $4 \times 2 + 1 = 8 + 1 = 9$~cm. Par les deux extrémités de ce segment on trace deux arcs de cercle de rayon 9 (cm) qui se coupent au troisième sommet du triangle équilatéral.
\item
	\begin{enumerate}
		\item Le périmètre du rectangle est égal à : $2(L + l) = 2(4x + 1,5 + 2x) = 2(6x + 1,5) = 12x + 3$.
		\item Il faut résoudre l'équation : 
        
        $12x + 3 = 18$ : On ajoute $- 3$ de chaque côté.
		
		$12x = 15$ : On divise par 12.
		
		$x = \dfrac{15}{12}$ : On simplifie.
		
		$x = \dfrac{5}{4}$
	\end{enumerate}
\item Le périmètre du triangle équilatéral est égal à : $3 \times (4x + 1) = 3 \times 4x + 3 \times 1 = 12x + 3$.

Quel que soit le nombre positif $x$, le triangle équilatéral et le rectangle ont le même périmètre.
\end{enumerate}

\textbf{Partie II}

A = 2 (on trace deux fois la longueur puis la largeur)

B = 90 (mesures des  angles d'un rectangle)

C = 3 (tracé des trois côtés)

D = 120 (mesure en degré des trois angles d'un triangle équilatéral : 60).

Le premier script trace le rectangle et le second le triangle équilatéral.

\end{document}