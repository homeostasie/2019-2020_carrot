\documentclass[12pt]{article}
\usepackage{geometry} % Pour passer au format A4
\geometry{hmargin=1cm, vmargin=1cm} % 

% Page et encodage
\usepackage[T1]{fontenc} % Use 8-bit encoding that has 256 glyphs
\usepackage[english,french]{babel} % Français et anglais
\usepackage[utf8]{inputenc} 

\usepackage{lmodern}
\setlength\parindent{0pt}

% Graphiques
\usepackage{graphicx,float,grffile}
\usepackage{pgf,tikz}
\usetikzlibrary{arrows}

% Maths et divers
\usepackage{amsmath,amsfonts,amssymb,amsthm,verbatim}
\usepackage{multicol,multido,enumitem,url,eurosym,gensymb}

% Sections
\usepackage{sectsty} % Allows customizing section commands
\allsectionsfont{\centering \normalfont\scshape}

% Tête et pied de page

\usepackage{fancyhdr} 
\pagestyle{fancyplain} 

\fancyhead{} % No page header
\fancyfoot{}

\renewcommand{\headrulewidth}{0pt} % Remove header underlines
\renewcommand{\footrulewidth}{0pt} % Remove footer underlines

\newcommand{\horrule}[1]{\rule{\linewidth}{#1}} % Create horizontal rule command with 1 argument of height

%----------------------------------------------------------------------------------------
%	Début du document
%----------------------------------------------------------------------------------------

\begin{document}

%----------------------------------------------------------------------------------------
% RE-DEFINITION
%----------------------------------------------------------------------------------------
% MATHS
%-----------

\newtheorem{Definition}{Définition}
\newtheorem{Theorem}{Théorème}
\newtheorem{Proposition}{Propriété}

% MATHS
%-----------
\renewcommand{\labelitemi}{$\bullet$}
\renewcommand{\labelitemii}{$\circ$}
\newcommand{\Pointilles}[1][3]{%
  \multido{}{#1}{\makebox[\linewidth]{\dotfill}\\[\parskip]
}}

%----------------------------------------------------------------------------------------
%	Titre
%----------------------------------------------------------------------------------------

\setlength{\columnseprule}{1pt}

\textbf{Nom, Prénom :} \hspace{8cm} \textbf{Classe :} \hspace{3cm} \textbf{Date :}\\
\vspace{-0.2cm}
\begin{center}
  \textit{On vit de ce que l’on obtient. On construit sa vie sur ce que l’on donne.}  - \textbf{Winston Churchill}
\end{center}
\vspace{-0.2cm}


\subsection*{ex1 - Calculer}
\begin{multicols}{3}
\begin{enumerate}
\item[1a.] $\sqrt{\sqrt{25} + 5^2} = $ \dotfill\\
\item[1b.] $\sqrt{\dfrac{9}{10}} = $ \dotfill\\
\item[1c.] $\dfrac{56}{4 + 3^2 + 3} = $ \dotfill \\
\item[1d.] $ \sqrt{24 + 4^2} + \sqrt{3} = $ \dotfill\\
\item[1e.] $\dfrac{\sqrt{25}}{\sqrt{6}} = $ \dotfill \\
\item[1f.] $(2\sqrt{24})^2 = $ \dotfill 
\end{enumerate}
\end{multicols}

\subsection*{ex2 - DRILL - Calculer une longueur}
\textbf{Écrire le calcul et le résultat.}
  
  \begin{center}
  \definecolor{zzttqq}{rgb}{0.6,0.2,0}
  \begin{tikzpicture}[line cap=round,line join=round,>=triangle 45,x=1.0cm,y=1.0cm,scale=0.5]
  \clip(-1.73,-4.29) rectangle (25.18,5.94);
  \fill[color=zzttqq,fill=zzttqq,fill opacity=0.1] (-0.24,2.54) -- (4.28,2.54) -- (-0.22,4.76) -- cycle;
  \fill[color=zzttqq,fill=zzttqq,fill opacity=0.1] (6.86,2.48) -- (10.7,2.44) -- (10.72,5.14) -- cycle;
  \fill[color=zzttqq,fill=zzttqq,fill opacity=0.1] (13.18,5.06) -- (13.51,-1.46) -- (17.57,-1.24) -- cycle;
  \fill[color=zzttqq,fill=zzttqq,fill opacity=0.1] (0.55,0.81) -- (-0.21,-2.41) -- (7.23,-3.72) -- cycle;
  \fill[color=zzttqq,fill=zzttqq,fill opacity=0.1] (19.07,4.46) -- (23.57,4.43) -- (23.46,-4.05) -- cycle;
  \draw [color=zzttqq] (-0.24,2.54)-- (4.28,2.54);
  \draw [color=zzttqq] (4.28,2.54)-- (-0.22,4.76);
  \draw [color=zzttqq] (-0.22,4.76)-- (-0.24,2.54);
  \draw [color=zzttqq] (6.86,2.48)-- (10.7,2.44);
  \draw [color=zzttqq] (10.7,2.44)-- (10.72,5.14);
  \draw [color=zzttqq] (10.72,5.14)-- (6.86,2.48);
  \draw [color=zzttqq] (13.18,5.06)-- (13.51,-1.46);
  \draw [color=zzttqq] (13.51,-1.46)-- (17.57,-1.24);
  \draw [color=zzttqq] (17.57,-1.24)-- (13.18,5.06);
  \draw [color=zzttqq] (0.55,0.81)-- (-0.21,-2.41);
  \draw [color=zzttqq] (-0.21,-2.41)-- (7.23,-3.72);
  \draw [color=zzttqq] (7.23,-3.72)-- (0.55,0.81);
  \draw [color=zzttqq] (19.07,4.46)-- (23.57,4.43);
  \draw [color=zzttqq] (23.57,4.43)-- (23.46,-4.05);
  \draw [color=zzttqq] (23.46,-4.05)-- (19.07,4.46);
  \end{tikzpicture}
\end{center}

\begin{multicols}{5}
\begin{enumerate}
\item[2a.] \dotfill 
\item[2b.] \dotfill
\item[2c.] \dotfill 
\item[2d.] \dotfill 
\item[2e.] \dotfill 
\end{enumerate}
\end{multicols}
\begin{multicols}{5}
\begin{enumerate}
\item[] \dotfill 
\item[] \dotfill
\item[] \dotfill 
\item[] \dotfill 
\item[] \dotfill 
\end{enumerate}
\end{multicols}


\subsection*{ex3 - \textbf{RÉDIGER} - Rectangle ?}

\begin{enumerate}
  \item[5a.]Soit WEQ un triangle tel que : EQ = 9,1 cm, QW = 3,5 cm et EW = 8,4 cm.\\
  \textbf{Quelle est la nature du triangle WEQ ?} \\ \Pointilles[4]

  \item[5b.]Soit HZM un triangle tel que : MH = 9,6 cm, ZH = 11 cm et ZM = 14,8 cm. \\
  \textbf{Quelle est la nature du triangle HZM ?} \\ \Pointilles[4]
\end{enumerate}


\newpage

\paragraph{Le premier problème est obligatoire. Ensuite au choix, vous faites le problème 2 \textbf{OU} le problème 3. L'autre problème est à rendre en DM.}

\subsection*{Problème 1 - Câble}

Deux immeubles sont distants de 45m. 
  \begin{itemize}
  \item Le premier mesure 13m de haut.
  \item Le deuxième mesure 9m. 
  \end{itemize}

On souhaite tendre un câble entre leur toit. 

\begin{enumerate}
  \item[I a.] Représenter \textit{(proprement)} la situation.

  \item[I b.] Calculer la longueur de ce câble. \textbf{calculer et justifier}

  \item[I c.] Pour acheter le câble à Bricoman (magasin), deux formules sont possibles :  
  \begin{itemize}
  \item Le câble est vendu  au prix de 6 \euro \, le mètre. 
  \item Une bobine de 5 mètres est vendu 25 \euro \,(la bobine de 5m). 
  \end{itemize} 
  Quelle est la meilleure offre ? \textbf{calculer et justifier}
\end{enumerate}

\subsection*{Problème 2 - Nappe}

\textsc{M. Lafond} a dans son salon une belle table rectangulaire au dimension de 2m par 1,20m. \\
Un brin distrait, il se trompe et achète une nappe chez Gifi qui est ronde et de diamètre de 2.50m. \\
Il n'a pas le temps d'aller la faire changer... \\
La nappe recouvre-t-elle toute la table ?

\begin{enumerate}
  \item[II a.] Que doit-on calculer afin de répondre à cette question ?

  \item[II b.] Répondre à la question. \textbf{calculer et justifier}
\end{enumerate}

\subsection*{Problème 3 - Zut, mes clés}

\textsc{M. Lafond} à oublié les clés de sa maison et doit passer par la fenêtre ouverte du premier étage pour rentrer chez lui. \\
Il possède une échelle de 5.6m. Afin d'être stable et ne pas tomber, l'échelle doit être écartée du bord de 1.30m. 

\begin{enumerate}
\item[III a.] Représenter \textit{(proprement)} la situation.

\item[III b.] \textsc{M. Lafond} peut-il atteindre sans danger la fenêtre du premier étage situé à 5m du sol ?  \textbf{calculer et justifier}
\end{enumerate}

\newpage


\textbf{Nom, Prénom :} \hspace{8cm} \textbf{Classe :} \hspace{3cm} \textbf{Date :}\\
\vspace{-0.2cm}
\begin{center}
  \textit{On vit de ce que l’on obtient. On construit sa vie sur ce que l’on donne.}  - \textbf{Winston Churchill}
\end{center}
\vspace{-0.2cm}


\subsection*{ex1 - Calculer}
\begin{multicols}{3}
\begin{enumerate}
\item[1a.] $\sqrt{\sqrt{45} + 4^2} = $ \dotfill\\
\item[1b.] $\sqrt{\dfrac{8}{9}} = $ \dotfill\\
\item[1c.] $\dfrac{26}{4 + 6^2 + 3} = $ \dotfill \\
\item[1d.] $\sqrt{23 + 3^2} + \sqrt{6} = $ \dotfill\\
\item[1e.] $\dfrac{\sqrt{35}}{\sqrt{6}} = $ \dotfill \\
\item[1f.] $(2\sqrt{14})^2 = $ \dotfill 
\end{enumerate}
\end{multicols}

\subsection*{ex2 - DRILL - Calculer une longueur}
\textbf{Écrire le calcul et le résultat.}
  
  \begin{center}
  \definecolor{zzttqq}{rgb}{0.6,0.2,0}
  \begin{tikzpicture}[line cap=round,line join=round,>=triangle 45,x=1.0cm,y=1.0cm,scale=0.5]
  \clip(-1.73,-4.29) rectangle (25.18,5.94);
  \fill[color=zzttqq,fill=zzttqq,fill opacity=0.1] (-0.24,2.54) -- (4.28,2.54) -- (-0.22,4.76) -- cycle;
  \fill[color=zzttqq,fill=zzttqq,fill opacity=0.1] (6.86,2.48) -- (10.7,2.44) -- (10.72,5.14) -- cycle;
  \fill[color=zzttqq,fill=zzttqq,fill opacity=0.1] (13.18,5.06) -- (13.51,-1.46) -- (17.57,-1.24) -- cycle;
  \fill[color=zzttqq,fill=zzttqq,fill opacity=0.1] (0.55,0.81) -- (-0.21,-2.41) -- (7.23,-3.72) -- cycle;
  \fill[color=zzttqq,fill=zzttqq,fill opacity=0.1] (19.07,4.46) -- (23.57,4.43) -- (23.46,-4.05) -- cycle;
  \draw [color=zzttqq] (-0.24,2.54)-- (4.28,2.54);
  \draw [color=zzttqq] (4.28,2.54)-- (-0.22,4.76);
  \draw [color=zzttqq] (-0.22,4.76)-- (-0.24,2.54);
  \draw [color=zzttqq] (6.86,2.48)-- (10.7,2.44);
  \draw [color=zzttqq] (10.7,2.44)-- (10.72,5.14);
  \draw [color=zzttqq] (10.72,5.14)-- (6.86,2.48);
  \draw [color=zzttqq] (13.18,5.06)-- (13.51,-1.46);
  \draw [color=zzttqq] (13.51,-1.46)-- (17.57,-1.24);
  \draw [color=zzttqq] (17.57,-1.24)-- (13.18,5.06);
  \draw [color=zzttqq] (0.55,0.81)-- (-0.21,-2.41);
  \draw [color=zzttqq] (-0.21,-2.41)-- (7.23,-3.72);
  \draw [color=zzttqq] (7.23,-3.72)-- (0.55,0.81);
  \draw [color=zzttqq] (19.07,4.46)-- (23.57,4.43);
  \draw [color=zzttqq] (23.57,4.43)-- (23.46,-4.05);
  \draw [color=zzttqq] (23.46,-4.05)-- (19.07,4.46);
  \end{tikzpicture}
\end{center}

\begin{multicols}{5}
\begin{enumerate}
\item[2a.] \dotfill 
\item[2b.] \dotfill
\item[2c.] \dotfill 
\item[2d.] \dotfill 
\item[2e.] \dotfill 
\end{enumerate}
\end{multicols}
\begin{multicols}{5}
\begin{enumerate}
\item[] \dotfill 
\item[] \dotfill
\item[] \dotfill 
\item[] \dotfill 
\item[] \dotfill 
\end{enumerate}
\end{multicols}

\subsection*{ex3 - \textbf{RÉDIGER} - Rectangle ?}

\begin{enumerate}
  \item[5a.]Soit WEQ un triangle tel que : EQ = 10,1 cm, QW = 4,5 cm et EW = 8,4 cm.\\
  \textbf{Quelle est la nature du triangle WEQ ?} \\ \Pointilles[4]

  \item[5b.]Soit HZM un triangle tel que : MH = 8,6 cm, ZH = 12 cm et ZM = 15,8 cm. \\
  \textbf{Quelle est la nature du triangle HZM ?} \\ \Pointilles[4]
\end{enumerate}

\newpage

\paragraph{Le premier problème est obligatoire. Ensuite au choix, vous faites le problème 2 \textbf{OU} le problème 3. L'autre problème est à rendre en DM.}

\subsection*{Problème 1 - Câble}

Deux immeubles sont distants de 54m. 
  \begin{itemize}
  \item Le premier mesure 16m de haut.
  \item Le deuxième mesure 12m. 
  \end{itemize}

On souhaite tendre un câble entre leur toit. 

\begin{enumerate}
  \item[I a.] Représenter \textit{(proprement)} la situation.

  \item[I b.] Calculer la longueur de ce câble. \textbf{calculer et justifier}

  \item[I c.] Pour acheter le câble à Bricoman (magasin), deux formules sont possibles :  
  \begin{itemize}
  \item Le câble est vendu  au prix de 5 \euro \, le mètre. 
  \item Une bobine de 5 mètres est vendu 22 \euro \,(la bobine de 5m). 
  \end{itemize} 
  Quelle est la meilleure offre ? \textbf{calculer et justifier}
\end{enumerate}

\subsection*{Problème 2 - Nappe}

\textsc{M. Lafond} a dans son salon une belle table rectangulaire au dimension de 2m par 1,30m. \\
Un brin distrait, il se trompe et achète une nappe chez Gifi qui est ronde et de diamètre de 2.60m. \\
Il n'a pas le temps d'aller la faire changer... \\
La nappe recouvre-t-elle toute la table ?

\begin{enumerate}
  \item[II a.] Que doit-on calculer afin de répondre à cette question ?

  \item[II b.] Répondre à la question. \textbf{calculer et justifier}
\end{enumerate}

\subsection*{Problème 3 - Zut, mes clés}

\textsc{M. Lafond} à oublié les clés de sa maison et doit passer par la fenêtre ouverte du premier étage pour rentrer chez lui. \\
Il possède une échelle de 5.4m. Afin d'être stable et ne pas tomber, l'échelle doit être écartée du bord de 1.20m. 

\begin{enumerate}
\item[III a.] Représenter \textit{(proprement)} la situation.

\item[III b.] \textsc{M. Lafond} peut-il atteindre sans danger la fenêtre du premier étage situé à 5m du sol ?  \textbf{calculer et justifier}
\end{enumerate}

\end{document}
