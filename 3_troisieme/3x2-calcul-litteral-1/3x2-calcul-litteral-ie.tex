\documentclass[11pt]{article}
\usepackage{geometry} % Pour passer au format A4
\geometry{hmargin=1cm, vmargin=1cm} % 

% Page et encodage
\usepackage[T1]{fontenc} % Use 8-bit encoding that has 256 glyphs
\usepackage[english,french]{babel} % Français et anglais
\usepackage[utf8]{inputenc} 

\usepackage{lmodern}
\setlength\parindent{0pt}

% Graphiques
\usepackage{graphicx,float,grffile}

% Maths et divers
\usepackage{amsmath,amsfonts,amssymb,amsthm,verbatim}
\usepackage{multicol,enumitem,url,eurosym,gensymb}

% Sections
\usepackage{sectsty} % Allows customizing section commands
\allsectionsfont{\centering \normalfont\scshape}

% Tête et pied de page

\usepackage{fancyhdr} 
\pagestyle{fancyplain} 

\fancyhead{} % No page header
\fancyfoot{}

\renewcommand{\headrulewidth}{0pt} % Remove header underlines
\renewcommand{\footrulewidth}{0pt} % Remove footer underlines

\newcommand{\horrule}[1]{\rule{\linewidth}{#1}} % Create horizontal rule command with 1 argument of height

%----------------------------------------------------------------------------------------
%	Début du document
%----------------------------------------------------------------------------------------

\begin{document}

%----------------------------------------------------------------------------------------
% RE-DEFINITION
%----------------------------------------------------------------------------------------
% MATHS
%-----------

\newtheorem{Definition}{Définition}
\newtheorem{Theorem}{Théorème}
\newtheorem{Proposition}{Propriété}

% MATHS
%-----------
\renewcommand{\labelitemi}{$\bullet$}
\renewcommand{\labelitemii}{$\circ$}
%----------------------------------------------------------------------------------------
%	Titre
%----------------------------------------------------------------------------------------

\setlength{\columnseprule}{1pt}

\textbf{Nom, Prénom :} \hspace{8cm} \textbf{Classe :} \hspace{3cm} \textbf{Date :}\\
\vspace{-0.8cm}
\begin{center}
  \textit{If you do the work, you get rewarded. There are no shortcuts ibn life.}  - \textbf{Michael Jordan}
\end{center}
\vspace{-0.8cm}

\subsection*{Équations}
\begin{multicols}{3}\noindent
  \begin{enumerate}
  \item[a.)] $-2x + 5 = 9$
  \item[b.)] $x - 10 = -11$
  \item[c.)] $-3x - 2 = 3x + 34$
  \item[d.)] $4x + 10 = -4x + 18$
  \item[e.)] $-4(x - 1) = 2x + 16$
  \item[f.)] $-3x = 12$
  \item[g.)] $x + 10 = 18$
  \item[h.)] $3(x + 8) = -3x + 18$
  \item[i.)] $x + 3 = -6$
  \item[j.)] $-4(x + 2) = -3x - 3$
  \item[k.)] $-3x + 7 = 3x - 35$
  \item[l.)] $-x - 9 = -17$
  \item[m.)] $-3x - 9 = -x - 23$
  \item[n.)] $4(x + 8) = 68$
  \item[o.)] $4x + 2 = 3x - 4$
  \end{enumerate}
\end{multicols}

\vspace{-0.4cm}
\horrule{1px}
\vspace{-0.8cm}

\subsection*{Programme de calcul}
\fbox{\begin{minipage}{0.4\textwidth}
  {\fontfamily{lmtt}\selectfont 
    \begin{itemize}
      \item Prendre le nombre de départ.
      \item Ajouter 10.
      \item multiplier par 4.
      \item Retrancher 40.
    \end{itemize}
  }
\end{minipage}}

\begin{enumerate}
  \item Essayer le programme de calcul avec 1, 2, 4, 512 et 1234.
  \item Peut-on prévoir le résultat si on prend 5000 comme nombre de départ ?
  \item Dire ce que fait le programme de calcul en une phrase.
  \item Est-ce suffisant pour démontrer que cela est vrai de prendre des exemples ?
  \item On pose $x$ comme nombre de départ. Effectuer le programme de calcul avec x. Réduire l’expression afin de conclure votre démonstration.
\end{enumerate}

\vspace{1cm}
\horrule{1px}
\vspace{1cm}

\textbf{Nom, Prénom :} \hspace{8cm} \textbf{Classe :} \hspace{3cm} \textbf{Date :}\\
\vspace{-0.8cm}
\begin{center}
  \textit{If you do the work, you get rewarded. There are no shortcuts ibn life.}  - \textbf{Michael Jordan}
\end{center}
\vspace{-0.8cm}

\subsection*{Équations}
\begin{multicols}{3}\noindent
  \begin{enumerate}
  \item[a.)] $x + 1 = 9$
  \item[b.)] $-5(x + 3) = 0$
  \item[c.)] $-4x - 4 = 3x + 38$
  \item[d.)] $x + 19 = 29$
  \item[e.)] $2x - 8 = -4x - 14$
  \item[f.)] $-3(x + 8) = 32$
  \item[g.)] $-4(x + 8) = -52$
  \item[h.)] $4(x - 3) = 12$
  \item[i.)] $-x + 5 = 5x + 11$
  \item[j.)] $x - 6 = -12$
  \item[k.)] $-4(x + 4) = 0$
  \item[l.)] $3x + 1 = -14$
  \item[m.)] $-x - 7 = x - 11$
  \item[n.)] $x + 3 = 8$
  \item[o.)] $2(x - 4) = 4x - 2$
  \end{enumerate}
\end{multicols}

\vspace{-0.4cm}
\horrule{1px}
\vspace{-0.8cm}

\subsection*{Programme de calcul}
\fbox{\begin{minipage}{0.4\textwidth}
  {\fontfamily{lmtt}\selectfont 
    \begin{itemize}
      \item Prendre le nombre de départ.
      \item Ajouter 10.
      \item multiplier par 6.
      \item Retrancher 60.
    \end{itemize}
  }
\end{minipage}}

\begin{enumerate}
  \item Essayer le programme de calcul avec 1, 2, 5, 412 et 2345.
  \item Peut-on prévoir le résultat si on prend 5000 comme nombre de départ ?
  \item Dire ce que fait le programme de calcul en une phrase.
  \item Est-ce suffisant pour démontrer que cela est vrai de prendre des exemples ?
  \item On pose $x$ comme nombre de départ. Effectuer le programme de calcul avec x. Réduire l’expression afin de conclure votre démonstration.
\end{enumerate}

\newpage

\textbf{Nom, Prénom :} \hspace{8cm} \textbf{Classe :} \hspace{3cm} \textbf{Date :}\\
\vspace{-0.8cm}
\begin{center}
  \textit{If you do the work, you get rewarded. There are no shortcuts ibn life.}  - \textbf{Michael Jordan}
\end{center}
\vspace{-0.8cm}

\subsection*{Équations}
\begin{multicols}{3}\noindent
  \begin{enumerate}
  \item[a.)] $x + 1 = 3x + 13$
  \item[b.)] $x - 4 = 0$
  \item[c.)] $-2x + 3 = -1$
  \item[d.)] $-5(x + 3) = -x - 33$
  \item[e.)] $-2(x + 2) = 3x - 39$
  \item[f.)] $-3x - 5 = -2x - 9$
  \item[g.)] $-4x - 7 = 1$
  \item[h.)] $x - 3 = -12$
  \item[i.)] $2x - 1 = -2x + 27$
  \item[j.)] $2(x - 9) = -4x - 72$
  \item[k.)] $x + 10 = 1$
  \item[l.)] $x + 20 = 29$
  \item[m.)] $2x - 6 = x - 8$
  \item[n.)] $4(x + 8) = 3x + 17$
  \item[o.)] $-2(x + 8) = -3x + 10$
  \end{enumerate}
\end{multicols}

\vspace{-0.4cm}
\horrule{1px}
\vspace{-0.8cm}

\subsection*{Programme de calcul}
\fbox{\begin{minipage}{0.4\textwidth}
  {\fontfamily{lmtt}\selectfont 
    \begin{itemize}
      \item Prendre le nombre de départ.
      \item Ajouter 6.
      \item multiplier par 5.
      \item Retrancher 30.
    \end{itemize}
  }
\end{minipage}}

\begin{enumerate}
  \item Essayer le programme de calcul avec 1, 2, 6, 612 et 4567.
  \item Peut-on prévoir le résultat si on prend 5000 comme nombre de départ ?
  \item Dire ce que fait le programme de calcul en une phrase.
  \item Est-ce suffisant pour démontrer que cela est vrai de prendre des exemples ?
  \item On pose $x$ comme nombre de départ. Effectuer le programme de calcul avec x. Réduire l’expression afin de conclure votre démonstration.
\end{enumerate}

\vspace{1cm}
\horrule{1px}
\vspace{1cm}

\textbf{Nom, Prénom :} \hspace{8cm} \textbf{Classe :} \hspace{3cm} \textbf{Date :}\\
\vspace{-0.8cm}
\begin{center}
  \textit{If you do the work, you get rewarded. There are no shortcuts ibn life.}  - \textbf{Michael Jordan}
\end{center}
\vspace{-0.8cm}

\subsection*{Équations}
\begin{multicols}{3}\noindent
  \begin{enumerate}
  \item[a.)] $	x + 3 = 5$
  \item[b.)] $x - 10 = -2x - 34$
  \item[c.)] $	3x + 6 = 2x + 10$
  \item[d.)] $4(x + 1) = -12$
  \item[e.)] $3(x - 20) = 21$
  \item[f.)] $x + 4 = 3$
  \item[g.)] $-3x + 3 = -x - 11$
  \item[h.)] $-3(x + 2) = -7$
  \item[i.)] $x + 12 = 16$
  \item[j.)] $	-2x + 4 = -4x + 6$
  \item[k.)] $4x = -32$
  \item[l.)] $3(x + 1) = 18$
  \item[m.)] $x + 12 = 19$
  \item[n.)] $-2(x + 2) = 1$
  \item[o.)] $2x = -4$
  \end{enumerate}
\end{multicols}

\vspace{-0.4cm}
\horrule{1px}
\vspace{-0.8cm}

\subsection*{Programme de calcul}
\fbox{\begin{minipage}{0.4\textwidth}
  {\fontfamily{lmtt}\selectfont 
    \begin{itemize}
      \item Prendre le nombre de départ.
      \item Ajouter 5
      \item multiplier par 8.
      \item Retrancher 40.
    \end{itemize}
  }
\end{minipage}}

\begin{enumerate}
  \item Essayer le programme de calcul avec 1, 2, 4, 312 et 7654.
  \item Peut-on prévoir le résultat si on prend 5000 comme nombre de départ ?
  \item Dire ce que fait le programme de calcul en une phrase.
  \item Est-ce suffisant pour démontrer que cela est vrai de prendre des exemples ?
  \item On pose $x$ comme nombre de départ. Effectuer le programme de calcul avec x. Réduire l’expression afin de conclure votre démonstration.
\end{enumerate}

\newpage

\textbf{Nom, Prénom :} \hspace{8cm} \textbf{Classe :} \hspace{3cm} \textbf{Date :}\\
\vspace{-0.8cm}
\begin{center}
  \textit{If you do the work, you get rewarded. There are no shortcuts ibn life.}  - \textbf{Michael Jordan}
\end{center}
\vspace{-0.8cm}

\subsection*{Équations}
\begin{multicols}{3}\noindent
  \begin{enumerate}
  \item[a.)] $-x - 5 = -4x + 16$
  \item[b.)] $4x = -24$
  \item[c.)] $4(x + 6) = 2x + 26$
  \item[d.)] $-3x + 3 = 24$
  \item[e.)] $x + 17 = 10$
  \item[f.)] $x - 10 = -19$
  \item[g.)] $-x - 10 = 5x + 44$
  \item[h.)] $x + 3 = 4$
  \item[i.)] $-2(x - 9) = 47$
  \item[j.)] $2x - 9 = 5x + 18$
  \item[k.)] $-2(x - 4) = 22$
  \item[l.)] $	3(x - 4) = -21$
  \item[m.)] $x + 5 = -2$
  \item[n.)] $-4x - 4 = 28$
  \item[o.)] $-4(x - 5) = 3x + 69$
  \end{enumerate}
\end{multicols}

\vspace{-0.4cm}
\horrule{1px}
\vspace{-0.8cm}

\subsection*{Programme de calcul}
\fbox{\begin{minipage}{0.4\textwidth}
  {\fontfamily{lmtt}\selectfont 
    \begin{itemize}
      \item Prendre le nombre de départ.
      \item Ajouter 8.
      \item multiplier par 8.
      \item Retrancher 64.
    \end{itemize}
  }
\end{minipage}}

\begin{enumerate}
  \item Essayer le programme de calcul avec 1, 2, 4, 212 et 5765.
  \item Peut-on prévoir le résultat si on prend 5000 comme nombre de départ ?
  \item Dire ce que fait le programme de calcul en une phrase.
  \item Est-ce suffisant pour démontrer que cela est vrai de prendre des exemples ?
  \item On pose $x$ comme nombre de départ. Effectuer le programme de calcul avec x. Réduire l’expression afin de conclure votre démonstration.
\end{enumerate}

\vspace{1cm}
\horrule{1px}
\vspace{1cm}

\textbf{Nom, Prénom :} \hspace{8cm} \textbf{Classe :} \hspace{3cm} \textbf{Date :}\\
\vspace{-0.8cm}
\begin{center}
  \textit{If you do the work, you get rewarded. There are no shortcuts ibn life.}  - \textbf{Michael Jordan}
\end{center}
\vspace{-0.8cm}

\subsection*{Équations}
\begin{multicols}{3}\noindent
  \begin{enumerate}
  \item[a.)] $	4x - 10 = -3x + 32$
  \item[b.)] $3x - 4 = 4x - 7$
  \item[c.)] $-2x - 7 = -9$
  \item[d.)] $2x = -2$
  \item[e.)] $	-x - 1 = 2x + 23$
  \item[f.)] $-5(x - 2) = 2x - 53$
  \item[g.)] $	x - 12 = -8$
  \item[h.)] $4x + 10 = 5x + 12$
  \item[i.)] $-2(x - 2) = 18$
  \item[j.)] $4x - 1 = -x + 44$
  \item[k.)] $	3(x - 4) = 5x - 12$
  \item[l.)] $	4(x + 10) = 8$
  \item[m.)] $4(x + 8) = 60$
  \item[n.)] $-5x = -10$
  \item[o.)] $-3x - 3 = -24$
  \end{enumerate}
\end{multicols}

\vspace{-0.4cm}
\horrule{1px}
\vspace{-0.8cm}

\subsection*{Programme de calcul}
\fbox{\begin{minipage}{0.4\textwidth}
  {\fontfamily{lmtt}\selectfont 
    \begin{itemize}
      \item Prendre le nombre de départ.
      \item Ajouter 4
      \item multiplier par 11.
      \item Retrancher 44.
    \end{itemize}
  }
\end{minipage}}

\begin{enumerate}
  \item Essayer le programme de calcul avec 1, 2, 4, 312 et 7654.
  \item Peut-on prévoir le résultat si on prend 5000 comme nombre de départ ?
  \item Dire ce que fait le programme de calcul en une phrase.
  \item Est-ce suffisant pour démontrer que cela est vrai de prendre des exemples ?
  \item On pose $x$ comme nombre de départ. Effectuer le programme de calcul avec x. Réduire l’expression afin de conclure votre démonstration.
\end{enumerate}
\end{document}
