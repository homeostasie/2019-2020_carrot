\documentclass[11pt]{article}
\usepackage{geometry} % Pour passer au format A4
\geometry{hmargin=1cm, vmargin=1cm} % 

% Page et encodage
\usepackage[T1]{fontenc} % Use 8-bit encoding that has 256 glyphs
\usepackage[english,french]{babel} % Français et anglais
\usepackage[utf8]{inputenc} 

\usepackage{lmodern}
\setlength\parindent{0pt}

% Graphiques
\usepackage{graphicx,float,grffile}

% Maths et divers
\usepackage{amsmath,amsfonts,amssymb,amsthm,verbatim}
\usepackage{multicol,enumitem,url,eurosym,gensymb}
\usepackage{xlop}

% Sections
\usepackage{sectsty} % Allows customizing section commands
\allsectionsfont{\centering \normalfont\scshape}

% Tête et pied de page

\usepackage{fancyhdr} 
\pagestyle{fancyplain} 

\fancyhead{} % No page header
\fancyfoot{}

\renewcommand{\headrulewidth}{0pt} % Remove header underlines
\renewcommand{\footrulewidth}{0pt} % Remove footer underlines

\newcommand{\horrule}[1]{\rule{\linewidth}{#1}} % Create horizontal rule command with 1 argument of height

\newcommand{\tempsexo}[1]{\textit{\textbf{(#1)}}}
%----------------------------------------------------------------------------------------
%   Début du document
%----------------------------------------------------------------------------------------

\begin{document}

%----------------------------------------------------------------------------------------
% RE-DEFINITION
%----------------------------------------------------------------------------------------
% MATHS
%-----------

\newtheorem{Definition}{Définition}
\newtheorem{Theorem}{Théorème}
\newtheorem{Proposition}{Propriété}
\newtheorem{Exo}{Éxercice}

% MATHS
%-----------
\renewcommand{\labelitemi}{$\bullet$}
\renewcommand{\labelitemii}{$\circ$}
%----------------------------------------------------------------------------------------
%   Titre
%----------------------------------------------------------------------------------------

\setlength{\columnseprule}{1pt}

\section*{S5 : Correction - Semaine du 13/04 au 19/04}

\subsection*{Travail sur le chapitre - Arithmétiques}

\Exo{p42 ex13}\\

\begin{enumerate}
    \item[1.] Quels sont les deux plus grands diviseurs de 95 ? \\
    $95 = 5 \times 19$ \\
    Les diviseurs de 95 sont : 1, 5, 19 et 95. \\
    \textbf{Les deux plus grands sont 19 et 95.}\\ 
    \item[b.] Quels sont les deux plus petits diviseurs de 45 ? \\
    $45 = 3^2 \times 5$ \\
    Les diviseurs de 45 sont : 1, 3, 5, 9, 15 et 45.\\
    \textbf{Les deux plus petits sont 1 et 3.}
\end{enumerate}

Remarque : C'est également une bonne occasion pour utiliser la fonction Décomp de sa calculatrice. 

\Exo{p43 ex15}\\

\textit{Parmi les nombres suivants, quel est celui qui a le plus de diviseurs ? }

\begin{enumerate}
    \item[a.] 10 : Les diviseurs sont : 1,2,5 et 10. \textbf{10 a 4 diviseurs.}
    \item[b.] 11 : Les diviseurs sont : 1 et 11. \textbf{11 a 2 diviseurs.}
    \item[c.] 12 : Les diviseurs sont : 1,2,3,4,6 et 12. \textbf{10 a 6 diviseurs.}
    \item[d.] 15 : Les diviseurs sont : 1,3, 5 et 25. \textbf{15 a 4 diviseurs.}
    \item[e.] 25 : Les diviseurs sont : 1,5 et 25. \textbf{10 a 3 diviseurs.}    
    \item[f.] 35 : Les diviseurs sont : 1,5,7 et 35. \textbf{10 a 4 diviseurs.}   
\end{enumerate}
\textbf{12 a le plus grand nombre de diviseurs : 6.}

Remarque : C'est également une bonne occasion pour utiliser la fonction Décomp de sa calculatrice. 

\Exo{p43 ex22}\\

\begin{multicols}{2}
Barbe-noire et 300 hommes. Il y a 6850 pièces d'or.
On doit faire la division Euclidienne de 6850 par 300.

Le partage entre les hommes d'équipages est équitables mais Barbe-noire empoche un beau petit paquet de pièces d'or en plus. \columnbreak

\opidiv{6850}{300} \\

Cela signifie que $6850 = 300 \times 22 + 250$.

\end{multicols}

Remarque : C'est également une bonne occasion pour utiliser la fonction division Euclidienne : $\vdash$ de sa calculatrice. 
Q= 22 signifie que le diviseur (quotient) est 22 et R= 250 signifie que le reste est 250.

\Exo{p43 ex25}\\

Olivia a 420 bd. On cherche les diviseurs de 420 à la main... et on s'arrête "à la moitié" : $\sqrt{420}$

\begin{multicols}{2}

\begin{itemize}
\item $420 = 1 \times 420$
\item $420 = 2 \times 210$
\item $420 = 3 \times 140$
\item $420 = 4 \times 105$
\item $420 = 5 \times 84$
\item $420 = 6 \times 70$
\item $420 = 7 \times 60$
\item $420 = 8 \times ...$ // pas divisible par 8. on passe à 9.
\item $420 = 9 \times ...$ // pas divisible par 9. on passe à 10.
\item $420 = 10 \times 42$
\item $420 = 12 \times 35$
\item $420 = 14 \times 30$
\item $420 = 15 \times 28$
\item $420 = 20 \times 21$
\end{itemize}
\end{multicols}

\end{document}