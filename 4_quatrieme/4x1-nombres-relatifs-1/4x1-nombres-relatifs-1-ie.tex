\documentclass[11pt]{article}
\usepackage{geometry} % Pour passer au format A4
\geometry{hmargin=1cm, vmargin=1cm} % 

% Page et encodage
\usepackage[T1]{fontenc} % Use 8-bit encoding that has 256 glyphs
\usepackage[english,french]{babel} % Français et anglais
\usepackage[utf8]{inputenc} 

\usepackage{lmodern}
\setlength\parindent{0pt}

% Graphiques
\usepackage{graphicx,float,grffile,tikz}

% Maths et divers
\usepackage{amsmath,amsfonts,amssymb,amsthm,verbatim}
\usepackage{multicol,enumitem,url,eurosym,gensymb,multido}

% Sections
\usepackage{sectsty} % Allows customizing section commands
\allsectionsfont{\centering \normalfont\scshape}

% Tête et pied de page

\usepackage{fancyhdr} 
\pagestyle{fancyplain} 

\fancyhead{} % No page header
\fancyfoot{}

\renewcommand{\headrulewidth}{0pt} % Remove header underlines
\renewcommand{\footrulewidth}{0pt} % Remove footer underlines

\newcommand{\horrule}[1]{\rule{\linewidth}{#1}} % Create horizontal rule command with 1 argument of height

%----------------------------------------------------------------------------------------
%	Début du document
%----------------------------------------------------------------------------------------

\begin{document}

% MATHS
%-----------
\newcommand{\Pointilles}[1][3]{%
  \multido{}{#1}{\makebox[\linewidth]{\dotfill}\\[\parskip]
}}
%----------------------------------------------------------------------------------------
%	Titre
%----------------------------------------------------------------------------------------

\setlength{\columnseprule}{1pt}

\textbf{Nom, Prénom :} \hspace{8cm} \textbf{Classe :} \hspace{3cm} \textbf{Date :}\\

\begin{center}
  \textit{Les mathématiques ne sont une moindre immensité que la mer.}  - \textbf{Victor Hugo}
\end{center}


\subsection*{ex1 - Calcul}

\begin{multicols}{3}\noindent

  \begin{enumerate}
  \item $12 - 3 = \ldots\ldots$
  \item $-5 + \ldots\ldots = -11$
  \item $3 + \left( -1\right) = \ldots\ldots$
  \item $-2 + \ldots\ldots = -11$
  \item $-2 \times 2 = \ldots\ldots$
  \item $\ldots\ldots + 7 = 16$
  \item $\ldots\ldots \div 10 = 4$
  \item $5 \times \left( -10\right) = \ldots\ldots$
  \item $10 \div \left( -5\right) = \ldots\ldots$
  \item $\ldots\ldots \times 8 = -64$
  \item $42 \div 6 = \ldots\ldots$
  \item $11 - \ldots\ldots = 5$
  \item $\ldots\ldots \times \left( -1\right) = 3$
  \item $-30 \div 6 = \ldots\ldots$
  \item $0 - \ldots\ldots = 3$
  \item $40 \div 4 = \ldots\ldots$
  \item $10 + 3 = \ldots\ldots$
  \item $-19 - \left( -9\right) = \ldots\ldots$
  \item $-5 \times \ldots\ldots = -20$
  \item $7 - \ldots\ldots = 5$
  \end{enumerate}
  
  \end{multicols}

\vspace{-0.4cm}
\horrule{1px}
\vspace{-0.8cm}

\begin{multicols}{2}

  \subsection*{ex2 - Programme de calcul}

  \textit{(sur feuille)}
  \begin{itemize}
  \item Prendre le nombre de départ.
  \item Lui ajouter 10
  \item Multiplier par -10 
  \item Ajouter cinq fois le nombre de départ. 
  \end{itemize}

  Effectuer le programme de calcul avec les nombres suivants :

  \begin{multicols}{2}

    \begin{enumerate}
    \item $x = 2$
    \item $x = 12$
    \item $x = -6$
    \item $x = 0,5$
    \end{enumerate}

  \end{multicols}

  \subsection*{ex3 - calcul litteral}

  \textit{(sur feuille)}
  \begin{enumerate}
  \item On pose a = 12 et b = -4.
    \begin{enumerate}
    \item Calculer : $2 \times a \times b + 14$
    \item Calculer : $(a - b)\times b - a$
    \end{enumerate}

  \item On pose a = -20 et b = -6.
    \begin{enumerate}
    \item Calculer : $2 \times a \times b + 14$
    \item Calculer : $(a - b)\times b - a$
    \end{enumerate}
  \end{enumerate}

\end{multicols}

\vspace{-0.4cm}
\horrule{1px}
\vspace{-0.8cm}

\begin{multicols}{2}

  \subsection*{ex4 - calcul pyramide}

  Chaque case est la somme des deux en dessous. Remplir la pyramide.

  \begin{tikzpicture}[every node/.style={draw,minimum width=2cm,minimum height=1cm}]
    \draw (0,0)  node {\phantom{\hspace{2cm}}};
    \draw(-1,-1) node {\phantom{\hspace{2cm}}} ++(2,0) node {\phantom{\hspace{2cm}}};
    \draw(-2,-2) node {\phantom{\hspace{2cm}}} ++(2,0) node {\phantom{\hspace{2cm}}} ++(2,0) node {\phantom{\hspace{2cm}}};
    \draw(-3,-3) node {7} ++(2,0) node {-8} ++(2,0) node {$-6 - 6$} ++(2,0) node {$-5 \times 3$};
  \end{tikzpicture}

  \subsection*{ex5 - calcul de signe}
  \textit{(sur feuille)}\\
  Sachant que $a$ est positif et $b$ est négatif. Donner le signe des expressions suivantes. \textit{Justifier.}

  \begin{enumerate}
  \item  $-2 \times a \times b \times -20$
  \item $\dfrac{-2 \times a \times b}{10 \times b}$
  \end{enumerate}

\end{multicols}

\vspace{-0.4cm}
\horrule{1px}
\vspace{-0.8cm}

\begin{multicols}{2}

  \subsection*{ex6 - calcul carré magique}

  Un carré est dit magique si la somme de ses lignes, la somme de ses colonnes et la somme de ses ont la même valeur. 
  \begin{enumerate}
  \item  Dire si le carré est magique. \textit{(sur feuille)}

    \begin{center}
      \begin{tabular}{|c|c|c|} \hline
        3 & 202 & 104 \\ \hline
        204 & 103 &   2 \\ \hline
        102 &   4 & 203 \\ \hline
      \end{tabular}
    \end{center}


  \item  Remplir le carré est magique.
    \begin{center}
      \begin{tabular}{|c|c|c|} \hline
        8 & 5 & 6  \\ \hline
        & 1 &    \\ \hline
        &   &    \\ \hline
      \end{tabular}
    \end{center}

  \end{enumerate}

\end{multicols}

\newpage

\textbf{Nom, Prénom :} \hspace{8cm} \textbf{Classe :} \hspace{3cm} \textbf{Date :}\\

\begin{center}
  \textit{Les mathématiques ne sont une moindre immensité que la mer.}  - \textbf{Victor Hugo}
\end{center}


\subsection*{Calcul}
\begin{multicols}{3}\noindent
  \begin{enumerate}
  \item $-4 - \ldots\ldots = -1$
  \item $-10 \times 6 = \ldots\ldots$
  \item $\ldots\ldots + 2 = 5$
  \item $7 + \left( -8\right) = \ldots\ldots$
  \item $7 - 8 = \ldots\ldots$
  \item $5 + \ldots\ldots = 13$
  \item $-45 \div \left( -9\right) = \ldots\ldots$
  \item $-6 - \ldots\ldots = -8$
  \item $\ldots\ldots - \left( -5\right) = 7$
  \item $63 \div 7 = \ldots\ldots$
  \item $-4 + 4 = \ldots\ldots$
  \item $12 \div \ldots\ldots = -3$
  \item $-6 \times \left( -9\right) = \ldots\ldots$
  \item $1 + 2 = \ldots\ldots$
  \item $-10 \times \ldots\ldots = 100$
  \item $5 - \left( -5\right) = \ldots\ldots$
  \item $12 \div 4 = \ldots\ldots$
  \item $-8 \times \left( -8\right) = \ldots\ldots$
  \item $\ldots\ldots \div 5 = 4$
  \item $-7 \times \left( -7\right) = \ldots\ldots$
  \end{enumerate}
\end{multicols}

\vspace{-0.4cm}
\horrule{1px}
\vspace{-0.8cm}

\begin{multicols}{2}

  \subsection*{ex2 - Programme de calcul}

  \textit{(sur feuille)}
  \begin{itemize}
  \item Prendre le nombre de départ.
  \item Lui ajouter 10
  \item Multiplier par -10 
  \item Ajouter six fois le nombre de départ. 
  \end{itemize}

  Effectuer le programme de calcul avec les nombres suivants :

  \begin{multicols}{2}

    \begin{enumerate}
    \item $x = 2$
    \item $x = 14$
    \item $x = -8$
    \item $x = 0,4$
    \end{enumerate}

  \end{multicols}

  \subsection*{ex3 - calcul litteral}

  \textit{(sur feuille)}
  \begin{enumerate}
  \item On pose a = 14 et b = -5.
    \begin{enumerate}
    \item Calculer : $2 \times a \times b - 12$
    \item Calculer : $(a - b)\times b - a$
    \end{enumerate}

  \item On pose a = -12 et b = -4.
    \begin{enumerate}
    \item Calculer : $2 \times a \times b - 14$
    \item Calculer : $(a - b)\times b - a$
    \end{enumerate}
  \end{enumerate}

\end{multicols}

\vspace{-0.4cm}
\horrule{1px}
\vspace{-0.8cm}

\begin{multicols}{2}

  \subsection*{ex4 - calcul pyramide}

  Chaque case est la somme des deux en dessous. Remplir la pyramide.

  \begin{tikzpicture}[every node/.style={draw,minimum width=2cm,minimum height=1cm}]
    \draw (0,0)  node {\phantom{\hspace{2cm}}};
    \draw(-1,-1) node {\phantom{\hspace{2cm}}} ++(2,0) node {\phantom{\hspace{2cm}}};
    \draw(-2,-2) node {\phantom{\hspace{2cm}}} ++(2,0) node {\phantom{\hspace{2cm}}} ++(2,0) node {\phantom{\hspace{2cm}}};
    \draw(-3,-3) node {6} ++(2,0) node {-10} ++(2,0) node {$-8 - 8$} ++(2,0) node {$-5 \times 2$};
  \end{tikzpicture}

  \subsection*{ex5 - calcul de signe}
  \textit{(sur feuille)}\\
  Sachant que $a$ est négatif et $b$ est positif. Donner le signe des expressions suivantes. \textit{Justifier.}

  \begin{enumerate}
  \item  $-2 \times a \times b \times -20$
  \item $\dfrac{-2 \times a \times b}{10 \times b}$
  \end{enumerate}

\end{multicols}

\vspace{-0.4cm}
\horrule{1px}
\vspace{-0.8cm}

\begin{multicols}{2}

  \subsection*{ex6 - calcul carré magique}

  Un carré est dit magique si la somme de ses lignes, la somme de ses colonnes et la somme de ses ont la même valeur. 
  \begin{enumerate}
  \item  Dire si le carré est magique. \textit{(sur feuille)}

    \begin{center}
      \begin{tabular}{|c|c|c|} \hline
        3 & 202 & 104 \\ \hline
        204 & 105 &   2 \\ \hline
        102 &   3 & 204 \\ \hline
      \end{tabular}
    \end{center}


  \item  Remplir le carré est magique.
    \begin{center}
      \begin{tabular}{|c|c|c|} \hline
        5 & 12 & 10  \\ \hline
        & 9 &    \\ \hline
        &   &    \\ \hline
      \end{tabular}
    \end{center}

  \end{enumerate}

\end{multicols}



\end{document}
