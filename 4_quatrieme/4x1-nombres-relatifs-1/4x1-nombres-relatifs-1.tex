\documentclass[12pt]{article}
\usepackage{geometry} % Pour passer au format A4
\geometry{hmargin=1cm, vmargin=1cm} % 

% Page et encodage
\usepackage[T1]{fontenc} % Use 8-bit encoding that has 256 glyphs
\usepackage[english,french]{babel} % Français et anglais
\usepackage[utf8]{inputenc} 

\usepackage{lmodern}
\setlength\parindent{0pt}

% Graphiques
\usepackage{graphicx,float,grffile}

% Maths et divers
\usepackage{amsmath,amsfonts,amssymb,amsthm,verbatim}
\usepackage{multicol,enumitem,url,eurosym,gensymb}

% Sections
\usepackage{sectsty} % Allows customizing section commands
\allsectionsfont{\centering \normalfont\scshape}

% Tête et pied de page

\usepackage{fancyhdr} 
\pagestyle{fancyplain} 

\fancyhead{} % No page header
\fancyfoot{}

\renewcommand{\headrulewidth}{0pt} % Remove header underlines
\renewcommand{\footrulewidth}{0pt} % Remove footer underlines

\newcommand{\horrule}[1]{\rule{\linewidth}{#1}} % Create horizontal rule command with 1 argument of height

%----------------------------------------------------------------------------------------
%   Début du document
%----------------------------------------------------------------------------------------

\begin{document}

%----------------------------------------------------------------------------------------
% RE-DEFINITION
%----------------------------------------------------------------------------------------
% MATHS
%-----------

\newtheorem{Definition}{Définition}
\newtheorem{Theorem}{Théorème}
\newtheorem{Proposition}{Propriété}

% MATHS
%-----------
\renewcommand{\labelitemi}{$\bullet$}
\renewcommand{\labelitemii}{$\circ$}
%----------------------------------------------------------------------------------------
%   Titre
%----------------------------------------------------------------------------------------

\setlength{\columnseprule}{1pt}

\horrule{2px}
\section*{Chapitre 1 - Nombres relatifs}
\horrule{2px}

\subsection*{Calcul avec deux nombres}

Un nombre doit être compris avec le signe qui est devant... et non celui d'après. On a tendance à ne pas mettre le $(+)$ quand il est inutile.

\begin{itemize}
\item $2 - 3 = 2 + (-3) = -1$
\end{itemize}

Deux signes $-$ $-$ qui se suivent se transforment en $+$. Le produit de deux nombres négatifs est positif. Une fraction avec deux nombres négatifs est positive. On doit simplifier (au moins mentalement) s'il y a trop de signe de $-$. 

\begin{itemize}
\item $4- (-6) =  4 + 6 = 10$
\item $2 \times 3 = -2 \times  -3 = 6 $ 
\item $\dfrac{12}{4} = \dfrac{-12}{-4} = 3 $
\item $- (-12) = 12$
\end{itemize} 
  
Pour le sens, une soustraction correspond à un écart.

\begin{itemize}
\item $2 - 7$ est l'écart entre 2 et 7 donc 5.
\item $3 - (-6)$ est l'écart entre 3 et $-6$ donc 9.
\end{itemize}


\subsection*{Calculs à plus de deux nombres}

Il faut respecter les priorités de calcul tout en acceptant de \og déplacer les nombres \fg \textbf{avec leurs signes} afin d'être astucieux. Il est important de créer sa propre méthode. 

Dans des produits et des fractions, on compte le nombre de signe de $-$ puis on les simplifie deux par deux.

$\dfrac{2 \times -3 \times -4}{5 \times  -2} = \dfrac{2 \times 3 \times 4}{5 \times  2}$ , on simplifie deux par deux. Il reste un seul signe $-$. La fraction est négative.


\subsection*{Activités possibles}

\begin{itemize}
\item Calcul à trou : On cherche un nombre dans un calcul dont on connait le résultat. On peut aussi chercher l'operation manquante. 
\item Comprendre le signe d'un calcul sans effectuer les opérations. 
\item Carré magique : On remplit des tableaux selon certaines règles.
\item Calcul à lettre. On écrit un calcul avec des lettres. On remplace les lettres par des nombres.
\item Programme de calcul : On effectue une suite d'instruction écrite en français.
\end{itemize}

\end{document}