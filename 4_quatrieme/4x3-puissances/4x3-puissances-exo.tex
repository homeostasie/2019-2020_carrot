\documentclass[12pt]{article}
\usepackage{geometry} % Pour passer au format A4
\geometry{hmargin=1cm, vmargin=1cm} % 

% Page et encodage
\usepackage[T1]{fontenc} % Use 8-bit encoding that has 256 glyphs
\usepackage[english,french]{babel} % Français et anglais
\usepackage[utf8]{inputenc} 

\usepackage{lmodern}
\setlength\parindent{0pt}

% Graphiques
\usepackage{graphicx,float,grffile}

% Maths et divers
\usepackage{amsmath,amsfonts,amssymb,amsthm,verbatim}
\usepackage{multicol,enumitem,url,eurosym,gensymb}

% Sections
\usepackage{sectsty} % Allows customizing section commands
\allsectionsfont{\centering \normalfont\scshape}

% Tête et pied de page

\usepackage{fancyhdr} 
\pagestyle{fancyplain} 

\fancyhead{} % No page header
\fancyfoot{}

\renewcommand{\headrulewidth}{0pt} % Remove header underlines
\renewcommand{\footrulewidth}{0pt} % Remove footer underlines

\newcommand{\horrule}[1]{\rule{\linewidth}{#1}} % Create horizontal rule command with 1 argument of height

%----------------------------------------------------------------------------------------
%	Début du document
%----------------------------------------------------------------------------------------

\begin{document}

%----------------------------------------------------------------------------------------
% RE-DEFINITION
%----------------------------------------------------------------------------------------
% MATHS
%-----------

\newtheorem{Definition}{Définition}
\newtheorem{Theorem}{Théorème}
\newtheorem{Proposition}{Propriété}

% MATHS
%-----------
\renewcommand{\labelitemi}{$\bullet$}
\renewcommand{\labelitemii}{$\circ$}
%----------------------------------------------------------------------------------------
%	Titre
%----------------------------------------------------------------------------------------

\setlength{\columnseprule}{1pt}

\horrule{2px}
\section*{Chapitre 3 - Puissances}
\horrule{2px}

\subsection*{1 - Calculer}

\textit{utiliser correctement la calculatrice et s'habituer à avoir certain résultat écrit différement.}

\subsubsection*{Ex 1 : Calculer}

\begin{multicols}{3}
  \begin{itemize}
  \item[a =] $7^2 + 1 =  \dotfill $
  \item[b =] $10^{5+3} =  \dotfill $
  \item[c =] $3^4 \times 4^3 =  \dotfill $
  \item[d =] $(\sqrt{2})^{20} =  \dotfill $
  \item[e =] $1^0 \times 2^1 \times 3^2 =  \dotfill $
  \item[f =] $123^0 =  \dotfill $
  \end{itemize}

\end{multicols}

\subsubsection*{Ex 2 : Calculer une grande expression }

\textit{(sans se tromper, ni obtenir d'erreur.)}

\begin{multicols}{2}

  $G = \dfrac{0,35 \times 10^{-3} \times 2,7 \times 10^{2}}{900 \times (10^5)^2} =  \dotfill $\\
  $H = \dfrac{80 \times 10^{-10} \times 0,18 \times 10^{-3}}{3,6 \times (10^{-7})^5} =  \dotfill $ \\
  $I = \dfrac{1800 \times 10^{9} \times 3600 \times 10^{8}}{1,2 \times (10^{-7})^2} =  \dotfill $\\
  $J = \dfrac{490 \times 10^{3} \times 3,6 \times 10^{2}}{25,2 \times (10^{10})^5} =  \dotfill $ 

\end{multicols}

\subsection*{2 - Restituer}

\textit{Restituer le cours et utiliser les méthodes de simplification.}

\subsubsection*{Ex 3 : Comprendre la notation puissance}

Remplacer la notation puissance par autant de signes $\times$ qu'il le faut. 

\begin{multicols}{3}
  \begin{itemize}
  \item[a =] $7^2 =  \dotfill $
  \item[b =] $10^{8} =  \dotfill $
  \item[c =] $-3^4  =  \dotfill $
  \item[d =] $(-3)^{4} =  \dotfill $
  \item[e =] $\dfrac{1}{4^4} =  \dotfill $
  \item[f =] $ 5^{-3} =  \dotfill $
  \item[g =] $ \dfrac{2^4}{5^{3}} =  \dotfill $
  \item[h =] $ \dfrac{1}{10^{4}} =  \dotfill $
  \end{itemize}
\end{multicols}

\subsubsection*{Ex 4 : utiliser les règles de simplifications.}

Compléter par un nombre de la forme $a^n$ avec $a$ et $n$ entiers :

\begin{multicols}{4}
  \begin{enumerate}
  \item[1.] $(11^{10})^{8} = \dotfill$
  \item[2.] $5^{4}  \times  3^{4}  =  \dotfill$
  \item[3.] $5^{6} \times 5^{5} = \dotfill$
  \item[4.] $\dfrac{11^{11}}{11^{6}} = \dotfill$
  \item[5.] $\dfrac{11^{9}}{11^{6}} = \dotfill$
  \item[6.] $5^{6} \times 5^{3} = \dotfill$
  \item[7.] $5^{7}  \times  7^{7}  =  \dotfill$
  \item[8.] $(10^{10})^{7} = \dotfill$
  \item[9.] $\dfrac{5^{11}}{5^{4}} = \dotfill$
  \item[10.] $(11^{3})^{5} = \dotfill$
  \item[11.] $(9^{8})^{7} = \dotfill$
  \item[12.] $9^{5} \times 9^{2} = \dotfill$
  \item[13.] $7^{4}  \times  2^{4}  =  \dotfill$
  \item[14.] $3^{7}  \times  5^{7}  =  \dotfill$
  \item[15.] $5^{9} \times 5^{7} = \dotfill$
  \item[16.] $\dfrac{11^{11}}{11^{8}} = \dotfill$
  \end{enumerate}
\end{multicols}

\newpage

\subsection*{3 - La notation scientifique}

\subsubsection*{Ex 5 : Compléter par le nombre qui convient :}

\begin{multicols}{3}

  \begin{enumerate}
  \item[1.] $6{,}098 \times \dotfill = 0{,}000\,006\,098$
  \item[2.] $6\,602 = 6{,}602 \times \dotfill$
  \item[3.] $0{,}004\,027 = 4{,}027 \times \dotfill$
  \item[4.] $0{,}060\,04 = 6{,}004 \times \dotfill$
  \item[5.] $610{,}9 = 6{,}109 \times \dotfill$
  \item[6.] $0{,}150\,2 = 1{,}502 \times \dotfill$
  \item[7.] $0{,}000\,630\,7 = 6{,}307 \times \dotfill$
  \item[8.] $7{,}306 \times \dotfill = 7\,306$
  \item[9.] $6{,}034 \times \dotfill = 60\,340\,000$
  \item[10.] $50{,}09 = 5{,}009 \times \dotfill$
  \item[11.] $0{,}206\,9 = 2{,}069 \times \dotfill$
  \item[12.] $40\,170 = 4{,}017 \times \dotfill$
  \end{enumerate}
\end{multicols}

\subsection*{4 - Problèmes}

\subsubsection*{Problème 1 – Distance Professeurs - Mars}

La distance Terre-Mars est 76 millions de kilomètres. 
M. Lafond mesure 1m65.

\begin{itemize}
\item[1.] Combien faut-il de \textsc{M. Lafond} pour atteindre Mars ?
\item[2.] Il faut 40 milliards de \textsc{M. Knopfer} pour atteindre Mars. Quelle taille fait-il ?
\end{itemize}

\subsubsection*{Problème 2 – Casa de Papel}

\textit{\og El Professeur \fg{} } vient de dérober 2 milliards d’euros. \\
Les billets de banque ont une épaisseur de $80 \times 10^{-6} m$. (On dit 80 micromètres)

\begin{itemize}
\item[1.] Quelle hauteur atteindrait une pile de billets de banque de 50 \euro{} représentant cette somme ?
\item[2.] Quelle hauteur atteindrait une pile de billets de banque de 10 \euro{} représentant cette somme ?
\end{itemize}

\subsubsection*{Problème 3 - Distance Collège- Soleil}

La lumière se propage à une vitesse de $3 \times 10^8 m/s$. \\
Un rayon partant du Soleil arrive au collège Frédéric Mistral au bout de $8 min 20s$.

\begin{itemize}
\item[1.] Convertir $8min 20s$ en un temps uniquement en seconde.
\item[2.] Quelle est la distance Collège- Soleil ? 
\end{itemize}

\subsubsection*{Problème 4 - Étoiles Vs Grain de sables.}

Il y a $2^{37}$ galaxies dans notre univers. Chaque galaxies contient $2^40$ étoiles.  \\
On estime le volume de sable sur Terre à $1\,000 \text{ milliards de } m^3$. Chaque $m^3$ contient environ $100 \text{milliards}$ de grains de sable. 

Inès affirme qu'il y a autant de grain de sable sur Terre que d'étoiles dans l'univers. A-t-elle raison ? 

\subsubsection*{Problème 5 - La légende de l'échiquier}

\og On place un grain de riz sur la première case d'un échiquier. Si on fait en sorte de doubler à chaque case le nombre de grains de la case précédente : un grain sur la première case, deux sur la deuxième, quatre sur la troisième, etc.,

\begin{itemize}
\item[1.] Sachant qu'il y a 64 cases, combien de grains de riz obtient-on sur la dernière case ? \fg
\item[2.] On estime que le nombre de grain de riz total sur l'échiquier est : $T = 2^{64} - 1$. \\
Faire ce calcul. La soustraction est-elle effectuée par la calculatrice ?
\end{itemize}


\end{document}
